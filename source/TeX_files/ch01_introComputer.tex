\chapter{Introduzione all'informatica e al computer}
\section{Definizione di informatica}
\textit{Informàtica} s. f. [dal fr. informatique, comp. di informat(ion) e (automat)ique «informazione automatica»], definita dalla Treccani come l’insieme dei vari aspetti scientifici e tecnici che sono specificamente applicati alla raccolta e al trattamento dell’informazione e in partica all’elaborazione automatica dei dati.\\
\section{Elementi di base}
Ecco una breve lista degli elemnti basilari di questa scienza:
\begin{itemize}
	\item\textbf{Algoritmo}: un algoritmo è una sequenza precisa (=deterministica) e finita di operazioni, comprensibili da un esecutore, che portano alla realizzazione di un compito.
	Le proprietà fondamentali di un algoritmo sono \textit{correttezza} ed \textit{efficienza}. Quando si opera con i calcolatori, gli algoritmi sono descritti utilizzando linguaggi che essi comprendono, i linguaggi di programmazione. 

	\item\textbf{Linguaggio di programmazione}: deve essere deterministico e rigoroso, esso è caratterizzato da: 
	\begin{itemize}
	    \item\textit{Sintassi}, cioè le regole che descrivono le stringhe di parole riconosciute dal linguaggio.
	    \item\textit{Semantica}, ovvero le regole per interpretazione delle stringhe e che descrivono i processi computazionali dell’ esecutore.
	\end{itemize}
    I linguaggi di programmazione si possono definire di alto o basso livello a seconda della loro maggiore vicinanza al linguaggio naturale (facilmente interpretabile dagli umani) o a quello della macchina (facilmente interpretabile dai calcolatori).

	\item\textbf{Programma}: è un algoritmo codificato tramite uno specifico linguaggio. 
	Data la complessità di alcuni programmi si sono sviluppati linguaggi intermedi (di alto livello) più vicini al linguaggio naturale che facilitano la scrittura dei programmi e che poi possono essere tradotti in linguaggi di basso livello, ovvero più vicini alla macchina (ed interpretabili da essa); esempi sono lo pseudocodice ed i diagrammi di flusso.
\end{itemize}

\section{}


