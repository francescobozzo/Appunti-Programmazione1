\chapter{Stringhe}

\section{Il concetto di stringa}
Come già visto nel capitolo \ref{vettori}, le stringhe sono strettamente legate al concetto di array, precisamente queste altro non sono che array di elementi di tipo char nella cui ultima cella è contenuto il carattere "$\backslash0$", identificato \textit{terminatore di stringa}.\\
Essendo usate comunemente, le stringhe hanno una serie di comandi che ne semplificano l'input e l'output (utilizzando il descrittore di formato \colorbox{light-gray}{\%s}).

\section{Operazioni con le stringhe}
Oltre a questo esiste una libreria standard interamente dedicata, \colorbox{light-gray}{string.h}, che comprende alcune utili funzioni quali \colorbox{light-gray}{strcpy()}, \colorbox{light-gray}{strlen()} e \colorbox{light-gray}{strcmp()}:
\begin{lstlisting}[title={Alcune operazione con le stringhe}]
char str1[5], str2[5];

fflush(stdin); // per Windows, da utilizzare prima di qualsiasi lettura di stringhe
scanf("%s", str1);
printf("%s", str1);

strcpy(str1, "cane"); //copia in str1 la stringa "cane"
strcpy(str2, str1); // copia il contenuto di str1 in str2

int lunghezza = strlen(str1); // restituisce la lunghezza di str1

bool uguali = strcmp(str1, "cane") == 0; // restituisce zero se uguali, >0 se str1>cane, <0 altrimenti
\end{lstlisting}
\section{Nota tecnica}
Considerando che l'ultimo carattere di una stringa deve essere il terminatore "$\backslash0$", e per scrivere una parola di \textit{n} lettere, bisogna utilizzare un array di lunghezza \textit{n+1}.
