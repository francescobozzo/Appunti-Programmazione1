\chapter{Stringhe}

\section{Il concetto di stringa}
Come già visto nel capitolo "vettori e matrici" le stringhe sono strettamente legate al concetto di array, precisamente queste altro non sono che array di elementi di tipo char nalla cui ultima cella è contenuto il carattere "$\backslash0$" che viene identificato terminatore di stringa.\\Le stringhe, essendo usate comunemente, hanno una serie di comandi che ne semplificano il trattamento, come ad esempio il fatto che si possa stampare una stringa (utilizzando l'identificatore \%s nella printf) senza dover implementare il classico ciclo che itera lungo tutto l'array stampandolo carattere per carattere. 
\section{Comandi dedicati alle stringhe}
Oltre a questo esiste una libreria standard interamente dedicata, <string.h>, che comprende comandi utilissimi quali \textbf{strcpy()} e \textbf{strlen()} il cui funzionamento è illustrato nell'esempio sottostante:
\begin{lstlisting}[title={Comandi comodi per trattare le stringhe}]
char str1[5], str2[5];
//esempi di printf e scanf semplificati
scanf("%s", &str1);
//copia direttmente in str1 il contenuto di stdin
printf("%s", str1);
//stampa direttamente la stringa str1 completa

strcpy(str1, "cane");
//copia in str1 la stringa cane e aggiunge '\0'
strcpy(str2, str1);
//copia il contenuto di str2 in str1 ('\0' compreso)
int lunghezza;
lunghezza=strlen(str1);
//restituisce la lunghezza di str1
\end{lstlisting}
\section{Nota tecnica}
Ricorda sempre che l'ultimo carattere di una stringa deve essere il terminatore "$\backslash0$" e quindi per scrivere una parola di \textit{n} lettere necessiteremo di un array di lunghezza \textit{n+1}.