\chapter{Liste}
Una \textbf{lista} è una struttura di dati astratta che permette di contenere un insieme di elementi. Attraverso la gestione dinamica della memoria non è inoltre necessario specificare la dimensione della struttura al momento della compilazione (attraverso costanti o \textit{define}).
Principalmente è possibile implementare le liste in due modi:
\begin{itemize}
	\item \textbf{semplicemente concatenate (sc)}: ogni elemento ha un puntatore all'elemento successivo
	\item \textbf{doppiamente concatenata (dc)}: ogni elemento ha due puntatori: uno all'elemento precedente e l'altro all'elemento successivo
\end{itemize}

Seppur la presenza di un ulteriore puntatore nell'implementazione \textit{dc} possa sembrare un'inutile ridondanza, in realtà permette di semplificare e ottimizzare alcuni algoritmi.\\
Entrambe le strutture presentano un metodo di accesso \textit{sequenziale} e non casuale (tipico degli array).

\begin{lstlisting}[title={Implementazione lista singolarmente concatenata}]
typedef struct Tnodo{
    Tdato dato;
    Tnodo *next;

    Tnodo(Tdato d);
    Tnodo(Tdato d, Tnodo *n);
} Tnodo;

\end{lstlisting}

\begin{lstlisting}[title={Implementazione lista doppiamente concatenata}]
struct Tnodo {
    Tdato dato;
    Tnodo *next;
    Tnodo *prev;
};
typedef Tnodo *ListaDoppiamenteConcatenata;
\end{lstlisting}